\documentclass[a4paper,12pt]{article}
\usepackage[utf8]{inputenc} % Soporte para caracteres UTF-8
\usepackage{graphicx} % Para insertar imágenes
\usepackage{listings} % Para código fuente
\usepackage{caption} % Para personalizar pies de imagen
\usepackage{longtable} % Para tablas largas
\usepackage{hyperref} % Para enlaces
\usepackage{fancyhdr} % Para encabezados y pies de página

% Configuración de encabezado y pie de página
\pagestyle{fancy}
\fancyhf{}
\rfoot{Texto en el pie de página}

\title{Título del Trabajo}
\author{Nombre del Autor\\ \texttt{correo@ejemplo.com}}
\date{Fecha}

\begin{document}

\maketitle

\section{Introducción}
Texto de introducción.

\section{Configuración del router en IPv6}

\subsection{Comandos de configuración}
Para configurar un router en IPv6, usamos los siguientes comandos:

\begin{lstlisting}[language=bash]
Router(config)# ipv6 unicast-routing
Router(config)# interface g0/0
Router(config-if)# ipv6 enable
Router(config-if)# ipv6 address FE80::1 link-local
Router(config-if)# ipv6 address 2001:db8:acad:1::1/64
Router(config-if)# no shutdown
\end{lstlisting}

\subsection{Inserción de imágenes}
La figura \ref{fig:red_local} muestra un esquema de red.

\begin{figure}[h]
    \centering
    \includegraphics[width=0.8\textwidth]{imagen.png}
    \caption{Esquema de la red}
    \label{fig:red_local}
\end{figure}

\section{Tablas}
Las direcciones IPv6 de la red se presentan en la tabla \ref{tab:ipv6_addresses}.

\begin{table}[h]
    \centering
    \begin{tabular}{|c|c|c|}
        \hline
        Dispositivo & Interfaz & Dirección IPv6 \\
        \hline
        R0 & G0/0 & 2001:db8:acad:1::1/64 \\
        PC0 & FastEthernet0 & 2001:db8:acad:1::A/64 \\
        \hline
    \end{tabular}
    \caption{Direcciones IPv6 asignadas}
    \label{tab:ipv6_addresses}
\end{table}

\section{Listas}
\subsection{Lista numerada}
\begin{enumerate}
    \item Primer ítem.
    \item Segundo ítem.
    \item Tercer ítem.
\end{enumerate}

\subsection{Lista con viñetas}
\begin{itemize}
    \item Primer punto.
    \item Segundo punto.
    \item Tercer punto.
\end{itemize}

\section{Referencias}
Para más información, ver \cite{ref1}.

\begin{thebibliography}{9}
    \bibitem{ref1} Autor, \textit{Título del libro o artículo}. Editorial, Año.
\end{thebibliography}

\end{document}
